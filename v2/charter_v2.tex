\documentclass[
11pt, % The default document font size, options: 10pt, 11pt, 12pt
%codirector, % Uncomment to add a codirector to the title page
]{charter} 




% El títulos de la memoria, se usa en la carátula y se puede usar el cualquier lugar del documento con el comando \ttitle
\titulo{Segmentación semántica en imágenes de microscopía utilizando redes neuronales} 

% Nombre del posgrado, se usa en la carátula y se puede usar el cualquier lugar del documento con el comando \degreename
%\posgrado{Carrera de Especialización en Sistemas Embebidos} 
%\posgrado{Carrera de Especialización en Internet de las Cosas} 
\posgrado{Carrera de Especialización en Inteligencia Artificial}
%\posgrado{Maestría en Sistemas Embebidos} 
%\posgrado{Maestría en Internet de las cosas}

% Tu nombre, se puede usar el cualquier lugar del documento con el comando \authorname
\autor{Lic. Pedro Miguel Pérez} 

% El nombre del director y co-director, se puede usar el cualquier lugar del documento con el comando \supname y \cosupname y \pertesupname y \pertecosupname
\director{Nombre del Director}
\pertenenciaDirector{pertenencia} 
% FIXME:NO IMPLEMENTADO EL CODIRECTOR ni su pertenencia
\codirector{John Doe} % para que aparezca en la portada se debe descomentar la opción codirector en el documentclass
\pertenenciaCoDirector{FIUBA}

% Nombre del cliente, quien va a aprobar los resultados del proyecto, se puede usar con el comando \clientename y \empclientename
\cliente{Mara Alderete}
\empresaCliente{Instituto de Nanosistemas UNSAM}

% Nombre y pertenencia de los jurados, se pueden usar el cualquier lugar del documento con el comando \jurunoname, \jurdosname y \jurtresname y \perteunoname, \pertedosname y \pertetresname.
\juradoUno{Nombre y Apellido (1)}
\pertenenciaJurUno{pertenencia (1)} 
\juradoDos{Nombre y Apellido (2)}
\pertenenciaJurDos{pertenencia (2)}
\juradoTres{Nombre y Apellido (3)}
\pertenenciaJurTres{pertenencia (3)}
 
\fechaINICIO{24 de junio de 2021}		%Fecha de inicio de la cursada de GdP 
\fechaFINALPlan{18 de agosto de 2021} 	%Fecha de final de cursada de GdP
\fechaFINALTrabajo{15 de mayo de 2022}	%Fecha de defensa pública del trabajo final


\begin{document}

\maketitle
\thispagestyle{empty}
\pagebreak


\thispagestyle{empty}
{\setlength{\parskip}{0pt}
\tableofcontents{}
}
\pagebreak


\section*{Registros de cambios}
\label{sec:registro}


\begin{table}[ht]
\label{tab:registro}
\centering
\begin{tabularx}{\linewidth}{@{}|c|X|c|@{}}
\hline
\rowcolor[HTML]{C0C0C0} 
Revisión & \multicolumn{1}{c|}{\cellcolor[HTML]{C0C0C0}Detalles de los cambios realizados} & Fecha      \\ \hline
0      & Creación del documento                                 &\fechaInicioName \\ \hline
1      & Se completa hasta el punto 5 inclusive                 & 08 de junio de 2021       \\ \hline
%2      & Se completa hasta el punto 7 inclusive
%		  Se puede agregar algo más \newline
%		  En distintas líneas \newline
%		  Así                                                    & dd/mm/aaaa \\ \hline
%3      & Se completa hasta el punto 11 inclusive                & dd/mm/aaaa \\ \hline
%4      & Se completa el plan	                                 & dd/mm/aaaa \\ \hline
\end{tabularx}
\end{table}

\pagebreak



\section*{Acta de constitución del proyecto}
\label{sec:acta}

\begin{flushright}
Buenos Aires, \fechaInicioName
\end{flushright}

\vspace{2cm}

Por medio de la presente se acuerda con el \authorname\hspace{1px} que su Trabajo Final de la \degreename\hspace{1px} se titulará ``\ttitle'' que consistirá esencialmente en la implementación de una red neuronal para el procesamiento de imágenes de microscopía, y tendrá un presupuesto preliminar estimado de 600 hs de trabajo, con fecha de inicio \fechaInicioName\hspace{1px} y fecha de presentación pública \fechaFinalName.

Se adjunta a esta acta la planificación inicial.

\vfill

% Esta parte se construye sola con la información que hayan cargado en el preámbulo del documento y no debe modificarla
\begin{table}[ht]
\centering
\begin{tabular}{ccc}
\begin{tabular}[c]{@{}c@{}}Ariel Lutenberg \\ Director posgrado FIUBA\end{tabular} & \hspace{2cm} & \begin{tabular}[c]{@{}c@{}}\clientename \\ \empclientename \end{tabular} \vspace{2.5cm} \\ 
\multicolumn{3}{c}{\begin{tabular}[c]{@{}c@{}} \supname \\ Director del Trabajo Final\end{tabular}} \vspace{2.5cm} \\
%\begin{tabular}[c]{@{}c@{}}\jurunoname \\ Jurado del Trabajo Final\end{tabular}     &  & \begin{tabular}[c]{@{}c@{}}\jurdosname\\ Jurado del Trabajo Final\end{tabular}  \vspace{2.5cm}  \\
%\multicolumn{3}{c}{\begin{tabular}[c]{@{}c@{}} \jurtresname\\ Jurado del Trabajo Final\end{tabular}} \vspace{.5cm}                                                                     
\end{tabular}
\end{table}




\section{1. Descripción técnica-conceptual del proyecto a realizar}
\label{sec:descripcion}
%\begin{consigna}{black} % El bloque "consigna" se usa para poner texto en rojo y dar una pequeña ayuda sobre cómo completar la sección
%Puede ser útil incluir en esta sección la respuesta a alguna de estas preguntas:
%
%\begin{itemize}
%	\item ¿Cómo se vincula este proyecto con la misión de la organización?
%	\item ¿Cómo se inserta este proyecto en el modelo de negocio de la organización?
%	\item ¿Ayuda a la explicación si se incluye un lienzo Canvas del Modelo de Negocio?
%	\item ¿En qué estado del ciclo de vida está el producto que se desea reemplazar o mejorar?
%	\item ¿Cuáles son las necesidades que debe satisfacer?
%	\item ¿Por dónde pasa la innovación?
%\end{itemize}

%La descripción técnica-conceptual \textbf{debe incluir al menos un diagrama en bloques del sistema} y una frase como la siguiente: ``En la Figura \ref{fig:diagBloques} se presenta el diagrama en bloques del sistema. Se observa que...''. Luego recién más abajo de haber puesto esta frase, se pone la figura. La regla es que las figuras nunca pueden ir antes de ser mencionadas en el texto, porque sino el lector no entiende por qué de pronto aparece una figura.
%
%
%El tamaño de la tipografía en TODAS las figuras debe ser adecuado para que NO pase lo que ocurre acá, donde el lector debe esforzarse para poder leer el texto. Los colores usados en el diagrama deben ser adecuados, tal que ayuden a comprender mejor el diagrama, preferentemente en la gama de colores pastel.
%
%\end{consigna}

Los esferoides son un modelo biológico de cultivo celular con estructura tridimensional. Los mismos son utilizados ampliamente en biología celular por poseer características similares al tejido del que provienen. Este tipo de cultivo constituye una herramienta atractiva dado que permite concluir que las condiciones de mantenimiento de las células en el cultivo son similares a las que se observan en un tejido de un organismo vivo.

Es posible generar esferoides mediante distintas metodologías, siendo una de las más utilizadas la denominada “gota pendiente”, que se basa en crear gotas de una suspensión celular sobre la cara interior de una superficie de vidrio o plástico. En cada gota creada las células forman un agregado con una estructura tridimensional similar a una esfera.

Las aplicaciones para las que han sido utilizados los esferoides son diversas, entre las que se destacan el screening de drogas con el objetivo de determinar aquellas más eficientes para combatir el crecimiento tumoral; la respuesta del esferoide al tratamiento se analiza midiendo dos parámetros predictivos claves: (I) variación de volumen y (II) cambios en la forma. Para ello, se suele registrar ambos indicadores a lo largo del tiempo realizando mediciones sobre imágenes obtenidas mediante microscopía de campo claro, lo cual conlleva un tiempo y esfuerzo considerable para el operador en el caso que realice el análisis de forma manual. 

Actualmente existe la necesidad de poder automatizar la medición de diferentes características en imágenes microscópicas de esferoides. El análisis manual realizado es propenso a errores, y a su vez añade el sesgo del operador que realiza la medición. Debido a esto, un método de procesamiento automático mejoraría la calidad y trazabilidad de los resultados y disminuiría considerablemente el tiempo de procesamiento. En la Figura \ref{fig:particulas} se muestra la secuencia de análisis y cuantificación de medidas en un esferoide. 

\begin{figure}[!hb]
\centering 
\includegraphics[width=.9\textwidth]{./Figuras/segmentacion_esferoides.png}
\caption{Secuencia de análisis de imágenes y cuantificación en un esferoide.}
\label{fig:particulas}
\end{figure}

Las disciplinas computacionales centradas en inteligencia artificial pueden aportar estrategias para facilitar el análisis y proponer soluciones a una variedad de problemas biológicos. Específicamente en este caso, contar con un algoritmo que pueda realizar este análisis automáticamente a partir de la segmentación de los esferoides presentes en las imágenes sería sumamente interesante para diversos grupos de investigación en este campo. Si bien se han realizado esfuerzos en la comunidad científica para avanzar con desarrollos en este sentido, los softwares generados no reconocen adecuadamente todos los formatos de imagen almacenados por distintos microscopios. Adicionalmente, la presencia de más de un esferoide por campo, o de bordes difusos en la imagen, generan dificultades para el adecuado funcionamiento de la rutina, introduciendo errores en la segmentación.

El objetivo de este trabajo es desarrollar un software capaz de segmentar y caracterizar automáticamente esferoides en cultivo a partir de imágenes obtenidas con microscopía de transmisión en campo claro, y así extraer medidas relevantes como área, radio e índice de esfericidad.


Las redes neuronales, en especial las redes neuronales convolucionales (Convolution Neural Network en inglés - CNN), han demostrado su amplio potencial en el procesamiento de imágenes para resolver problemas de clasificación y segmentación de objetos. El objetivo principal de este trabajo es poder aplicar técnicas de aprendizaje automático, en particular redes neuronales convolucionales, para procesar imágenes automáticamente y así poder segmentar y cuantificar esferoides en las mismas. En este trabajo el principal desafío es crear un sistema automático para el procesamiento de imágenes en un entorno real contando con un número limitado de imágenes para el entrenamiento.

El esquema de trabajo propuesto consiste en tres fases: generación del dataset, entrenamiento del modelo y aplicación del modelo. En la etapa de generación del dataset, a partir de un conjunto de imágenes reales se realizará un proceso de segmentación manual en la cual se extraerá el área de interés o ROI (region of interes en inglés), las imágenes segmentadas se separarán en dos sets, uno de ellos será utilizado para el entrenamiento y el otro para la evaluación del modelo. En la aplicación del modelo, se plantea una aplicación web que permitirá ingresar una imagen al modelo para lograr la predicción del área de interés o ROI del esferoide, a partir de esto se realizará una cuantificación de las medidas de interés, perímetro, radios e índice de esfericidad. En la Figura \ref{fig:diagBloques} se presenta el esquema de trabajo propuesto.





\begin{figure}[!hb]
\centering 
\includegraphics[width=1\textwidth]{./Figuras/diagrama.png}
\caption{Esquema de trabajo propuesto.}
\label{fig:diagBloques}
\end{figure}




\section{2. Identificación y análisis de los interesados}
\label{sec:interesados}

%Nota: (borrar esto y todas las consignas en color rojo antes de entregar este documento).
% 
%Es inusual que una misma persona esté en más de un rol, incluso en proyectos chicos.
% 
%Si se considera que una persona cumple dos o más roles, entonces sólo dejarla en el rol más importante. Por ejemplo:
%
%\begin{itemize}
%\item Si una persona es Cliente pero también colabora u orienta, dejarla solo como Cliente.
%\item Si una persona es el Responsable, no debe ser colocado también como Miembro del equipo.
%\end{itemize}
%
%Pero en cambio sí es usual que el Cliente y el Auspiciante sean el mismo, por ejemplo.

\begin{table}[ht]
%\caption{Identificación de los interesados}
%\label{tab:interesados}
\begin{tabularx}{\linewidth}{@{}|l|X|X|l|@{}}
\hline
\rowcolor[HTML]{C0C0C0} 
Rol           & Nombre y Apellido      & Organización 	& Puesto 	\\ \hline
Cliente       & Lic. Mara Alderete     &\empclientename	& Coordinadora UNSAM\\ \hline
Responsable   & \authorname            & FIUBA        	& Alumno 	\\ \hline
Colaboradores &  Lic. Marina Simian    & CONICET        &Investigadora \\ \hline
Orientador    & \supname	           & \pertesupname 	& Director Trabajo final \\ \hline
Usuario final &Operador                &UNSAM          	&Analista        	\\ \hline
\end{tabularx}
\end{table}

%El Director suele ser uno de los Orientadores.
%
%No dejar celdas vacías; si no hay nada que poner en una celda colocar un signo ``-''.
%
%No dejar filas vacías; si no hay nada que poner en una fila entonces eliminarla.
%
%Es deseable listar a continuación las principales características de cada interesado.
% 
%Por ejemplo:
\begin{itemize}
	\item Colaboradores: Marina Simian será de gran ayuda para la creación de los dataset necesarios para entrenar y validar el modelo.
	\item Usuario final: Los analistas del laboratorio serán de gran ayuda para validar el modelo en un entorno productivo.
\end{itemize}




\section{3. Propósito del proyecto}
\label{sec:proposito}

%\begin{consigna}{red}
%¿Por qué se hace el proyecto? ¿Qué se quiere lograr? 
%
%Se recomienda que sea solo un párrafo que empiece diciendo ``El propósito de este proyecto es...''.
%\end{consigna}
El propósito de este proyecto es diseñar e implementar una solución basada en una arquitectura de red neuronal convolucional que permita determinar características morfológicas aplicando técnicas de detección y segmentación eficiente en imágenes de microscopía.



\section{4. Alcance del proyecto}
\label{sec:alcance}

%\begin{consigna}{red}
%¿Qué se incluye y que no se incluye en este proyecto?
%
%Se refiere al trabajo a hacer para entregar el producto o resultado especificado. 
%
%Explicitar todo lo quede comprendido dentro del alcance del proyecto.
%
%Explicitar además todo lo que no quede incluido (``El presente proyecto no incluye...'')
%
%\end{consigna}

El desarrollo del proyecto incluye:
\begin{itemize}
	\item Desarrollar una herramienta que permita determinar el tamaño de esferoides en una imagen de microscopía utilizando redes neuronales convolucionales (CNN).
	\item Estudiar y comparar diferentes arquitecturas de redes neuronales para evaluar su rendimiento. Se plantea explorar las siguientes arquitecturas (U-NET, FastFCN y Mask R-CNN).
	\item Crear un dataset de imágenes que permita el entrenamiento de la red.
	\item Transformar y adaptar el dataset utilizado con el fin de incrementar el número de muestras y así poder aumentar la performance del modelo.
	\item Desarrollar una CNN que permita determinar la distribución del tamaño de partículas en dos clases de imágenes: esferoides aislados y aglomerado de los mismos.
	\item Evaluar la efectividad del modelo y aplicar técnicas que permitan mejorar el rendimiento de la red aplicando técnicas de optimización de los hiper parámetros.
	\item Escalar el modelo para el procesamiento de imágenes de naturaleza distinta a los esferoides, como por ejemplo imágenes de nanopartículas.
	\item Desarrollar un servicio web básico, donde se podrá hacer uso de la red neuronal para analizar las imágenes de esferoides.

\end{itemize}




\section{5. Supuestos del proyecto}
\label{sec:supuestos}

%\begin{consigna}{red}
%``Para el desarrollo del presente proyecto se supone que: ...''
%
%\begin{itemize}
%	\item Supuesto 1
%	\item Supuesto 2...
%\end{itemize}
%
%Por ejemplo, se podrían incluir supuestos respecto a disponibilidad de tiempo y recursos humanos y materiales, sobre la factibilidad técnica de distintos aspectos del proyecto, sobre otras cuestiones que sean necesarias para el éxito del proyecto como condiciones macroeconómicas o reglamentarias.
%\end{consigna}

Para el desarrollo del presente proyecto se supone que:
\begin{itemize}
	\item Se dispondrá un conjunto de imágenes previamente segmentadas para realizar el entrenamiento del modelo.
	\item El cliente deberá colaborar en la recolección y segmentación manual de las imágenes que serán utilizadas para el entrenamiento del modelo.
	\item El entrenamiento del modelo será realizado sobre la plataforma Colab.
	\
\end{itemize}

\section{6. Requerimientos}
\label{sec:requerimientos}

%\begin{consigna}{red}
%Los requerimientos deben numerarse y de ser posible estar agruparlos por afinidad, por ejemplo:
%
%\begin{enumerate}
%	\item Requerimientos funcionales
%		\begin{enumerate}
%			\item El sistema debe...
%			\item Tal componente debe...
%			\item El usuario debe poder...
%		\end{enumerate}
%	\item Requerimientos de documentación
%		\begin{enumerate}
%			\item Requerimiento 1
%			\item Requerimiento 2 (prioridad menor)
%		\end{enumerate}
%	\item Requerimiento de testing...
%	\item Requerimientos de la interfaz...
%	\item Requerimientos interoperabilidad...
%	\item etc...
%\end{enumerate}
%
%Leyendo los requerimientos se debe poder interpretar cómo será el proyecto y su funcionalidad.
%
%Indicar claramente cuál es la prioridad entre los distintos requerimientos y si hay requerimientos opcionales. 
%
%No olvidarse de que los requerimientos incluyen a las regulaciones y normas vigentes!!!
%
%Y al escribirlos seguir las siguientes reglas:
%\begin{itemize}
%	\item Ser breve y conciso (nadie lee cosas largas). 
%	\item Ser específico: no dejar lugar a confusiones.
%	\item Expresar los requerimientos en términos que sean cuantificables y medibles.
%\end{itemize}
%
%\end{consigna}


\begin{enumerate}
	\item Requerimientos funcionales
		\begin{enumerate}
			\item El sistema debe permitir seleccionar una imagen desde una interfaz web para ser procesada.
			\item El algoritmo debe detectar un esferoide dentro de la imagen procesada y mostrar la segmentación realizada.
			\item El algoritmo debe medir el esferoide detectado y brindar información del área, radio y perímetro.
			\item El sistema debe ser fácil de utilizar.
		\end{enumerate}
	\item Requerimientos no funcionales.
		\begin{enumerate}
			\item El sistema debe permitir procesar imágenes en diferentes formatos.
			\item El sistema debe poseer un rango de error menor al 5 por ciento con respecto al dataset de evaluación 
			\item El sistema debe poseer una interfaz web para procesar las imágenes.
			\item El sistema debe poder procesar una imagen en un rango de tiempo aceptable.
		\end{enumerate}
\end{enumerate}

\section{7. Historias de usuarios (\textit{Product backlog})}
\label{sec:backlog}

\begin{consigna}{red}
Descripción: En esta sección se deben incluir las historias de usuarios y su ponderación (\textit{history points}). Recordar que las historias de usuarios son descripciones cortas y simples de una característica contada desde la perspectiva de la persona que desea la nueva capacidad, generalmente un usuario o cliente del sistema. La ponderación es un número entero que representa el tamaño de la historia comparada con otras historias de similar tipo.

El formato propuesto es: "como [rol] quiero [tal cosa] para [tal otra cosa]."

Se debe indicar explícitamente el criterio para calcular los \textit{story points} de cada historia
\end{consigna}

\section{8. Entregables principales del proyecto}
\label{sec:entregables}

\begin{consigna}{red}

Los entregables del proyecto son (ejemplo):

\begin{itemize}
	\item Manual de uso
	\item Diagrama de circuitos esquemáticos
	\item Código fuente del firmware
	\item Diagrama de instalación
	\item Informe final
	\item etc...
\end{itemize}

\end{consigna}

\section{9. Desglose del trabajo en tareas}
\label{sec:wbs}

\begin{consigna}{red}
El WBS debe tener relación directa o indirecta con los requerimientos.  Son todas las actividades que se harán en el proyecto para dar cumplimiento a los requerimientos. Se recomienda mostrar el WBS mediante una lista indexada:

\begin{enumerate}
\item Grupo de tareas 1
	\begin{enumerate}
	\item Tarea 1 (tantas hs)
	\item Tarea 2 (tantas hs)
	\item Tarea 3 (tantas hs)
	\end{enumerate}
\item Grupo de tareas 2
	\begin{enumerate}
	\item Tarea 1 (tantas hs)
	\item Tarea 2 (tantas hs)
	\item Tarea 3 (tantas hs)
	\end{enumerate}
\item Grupo de tareas 3
	\begin{enumerate}
	\item Tarea 1 (tantas hs)
	\item Tarea 2 (tantas hs)
	\item Tarea 3 (tantas hs)
	\item Tarea 4 (tantas hs)
	\item Tarea 5 (tantas hs)
	\end{enumerate}
\end{enumerate}

Cantidad total de horas: (tantas hs)

Se recomienda que no haya ninguna tarea que lleve más de 40 hs. 

\end{consigna}

\section{10. Diagrama de Activity On Node}
\label{sec:AoN}

\begin{consigna}{red}
Armar el AoN a partir del WBS definido en la etapa anterior. 

%La figura \ref{fig:AoN} fue elaborada con el paquete latex tikz y pueden consultar la siguiente referencia \textit{online}:

%\url{https://www.overleaf.com/learn/latex/LaTeX_Graphics_using_TikZ:_A_Tutorial_for_Beginners_(Part_3)\%E2\%80\%94Creating_Flowcharts}

\end{consigna}

\begin{figure}[htpb]
\centering 
\includegraphics[width=.8\textwidth]{./Figuras/AoN.png}
\caption{Diagrama en \textit{Activity on Node}}
\label{fig:AoN}
\end{figure}

Indicar claramente en qué unidades están expresados los tiempos.
De ser necesario indicar los caminos semicríticos y analizar sus tiempos mediante un cuadro.
Es recomendable usar colores y un cuadro indicativo describiendo qué representa cada color, como se muestra en el siguiente ejemplo:



\section{11. Diagrama de Gantt}
\label{sec:gantt}

\begin{consigna}{red}

Existen muchos programas y recursos \textit{online} para hacer diagramas de gantt, entre los cuales destacamos:

\begin{itemize}
\item Planner
\item GanttProject
\item Trello + \textit{plugins}. En el siguiente link hay un tutorial oficial: \\ \url{https://blog.trello.com/es/diagrama-de-gantt-de-un-proyecto}
\item Creately, herramienta online colaborativa. \\\url{https://creately.com/diagram/example/ieb3p3ml/LaTeX}
\item Se puede hacer en latex con el paquete \textit{pgfgantt}\\ \url{http://ctan.dcc.uchile.cl/graphics/pgf/contrib/pgfgantt/pgfgantt.pdf}
\end{itemize}

Pegar acá una captura de pantalla del diagrama de Gantt, cuidando que la letra sea suficientemente grande como para ser legible. 
Si el diagrama queda demasiado ancho, se puede pegar primero la ``tabla'' del Gantt y luego pegar la parte del diagrama de barras del diagrama de Gantt.

Configurar el software para que en la parte de la tabla muestre los códigos del EDT (WBS).\\
Configurar el software para que al lado de cada barra muestre el nombre de cada tarea.\\
Revisar que la fecha de finalización coincida con lo indicado en el Acta Constitutiva.

En la figura \ref{fig:gantt}, se muestra un ejemplo de diagrama de gantt realizado con el paquete de \textit{pgfgantt}. En la plantilla pueden ver el código que lo genera y usarlo de base para construir el propio.

\begin{figure}[htbp]
\begin{center}
\begin{ganttchart}{1}{12}
  \gantttitle{2020}{12} \\
  \gantttitlelist{1,...,12}{1} \\
  \ganttgroup{Group 1}{1}{7} \\
  \ganttbar{Task 1}{1}{2} \\
  \ganttlinkedbar{Task 2}{3}{7} \ganttnewline
  \ganttmilestone{Milestone o hito}{7} \ganttnewline
  \ganttbar{Final Task}{8}{12}
  \ganttlink{elem2}{elem3}
  \ganttlink{elem3}{elem4}
\end{ganttchart}
\end{center}
\caption{Diagrama de gantt de ejemplo}
\label{fig:gantt}
\end{figure}


\begin{landscape}
\begin{figure}[htpb]
\centering 
\includegraphics[height=.85\textheight]{./Figuras/Gantt-2.png}
\caption{Ejemplo de diagrama de Gantt rotado}
\label{fig:diagGantt}
\end{figure}

\end{landscape}

\end{consigna}


\section{12. Presupuesto detallado del proyecto}
\label{sec:presupuesto}

\begin{consigna}{red}
Si el proyecto es complejo entonces separarlo en partes:
\begin{itemize}
	\item Un total global, indicando el subtotal acumulado por cada una de las áreas.
	\item El desglose detallado del subtotal de cada una de las áreas.
\end{itemize}

IMPORTANTE: No olvidarse de considerar los COSTOS INDIRECTOS.

\end{consigna}

\begin{table}[htpb]
\centering
\begin{tabularx}{\linewidth}{@{}|X|c|r|r|@{}}
\hline
\rowcolor[HTML]{C0C0C0} 
\multicolumn{4}{|c|}{\cellcolor[HTML]{C0C0C0}COSTOS DIRECTOS} \\ \hline
\rowcolor[HTML]{C0C0C0} 
Descripción &
  \multicolumn{1}{c|}{\cellcolor[HTML]{C0C0C0}Cantidad} &
  \multicolumn{1}{c|}{\cellcolor[HTML]{C0C0C0}Valor unitario} &
  \multicolumn{1}{c|}{\cellcolor[HTML]{C0C0C0}Valor total} \\ \hline
 &
  \multicolumn{1}{c|}{} &
  \multicolumn{1}{c|}{} &
  \multicolumn{1}{c|}{} \\ \hline
 &
  \multicolumn{1}{c|}{} &
  \multicolumn{1}{c|}{} &
  \multicolumn{1}{c|}{} \\ \hline
\multicolumn{1}{|l|}{} &
   &
   &
   \\ \hline
\multicolumn{1}{|l|}{} &
   &
   &
   \\ \hline
\multicolumn{3}{|c|}{SUBTOTAL} &
  \multicolumn{1}{c|}{} \\ \hline
\rowcolor[HTML]{C0C0C0} 
\multicolumn{4}{|c|}{\cellcolor[HTML]{C0C0C0}COSTOS INDIRECTOS} \\ \hline
\rowcolor[HTML]{C0C0C0} 
Descripción &
  \multicolumn{1}{c|}{\cellcolor[HTML]{C0C0C0}Cantidad} &
  \multicolumn{1}{c|}{\cellcolor[HTML]{C0C0C0}Valor unitario} &
  \multicolumn{1}{c|}{\cellcolor[HTML]{C0C0C0}Valor total} \\ \hline
\multicolumn{1}{|l|}{} &
   &
   &
   \\ \hline
\multicolumn{1}{|l|}{} &
   &
   &
   \\ \hline
\multicolumn{1}{|l|}{} &
   &
   &
   \\ \hline
\multicolumn{3}{|c|}{SUBTOTAL} &
  \multicolumn{1}{c|}{} \\ \hline
\rowcolor[HTML]{C0C0C0}
\multicolumn{3}{|c|}{TOTAL} &
   \\ \hline
\end{tabularx}%
\end{table}


\section{13. Gestión de riesgos}
\label{sec:riesgos}

\begin{consigna}{red}
a) Identificación de los riesgos (al menos cinco) y estimación de sus consecuencias:
 
Riesgo 1: detallar el riesgo (riesgo es algo que si ocurre altera los planes previstos de forma negativa)
\begin{itemize}
	\item Severidad (S): mientras más severo, más alto es el número (usar números del 1 al 10).\\
	Justificar el motivo por el cual se asigna determinado número de severidad (S).
	\item Probabilidad de ocurrencia (O): mientras más probable, más alto es el número (usar del 1 al 10).\\
	Justificar el motivo por el cual se asigna determinado número de (O). 
\end{itemize}   

Riesgo 2:
\begin{itemize}
	\item Severidad (S): 
	\item Ocurrencia (O):
\end{itemize}

Riesgo 3:
\begin{itemize}
	\item Severidad (S): 
	\item Ocurrencia (O):
\end{itemize}


b) Tabla de gestión de riesgos:      (El RPN se calcula como RPN=SxO)

\begin{table}[htpb]
\centering
\begin{tabularx}{\linewidth}{@{}|X|c|c|c|c|c|c|@{}}
\hline
\rowcolor[HTML]{C0C0C0} 
Riesgo & S & O & RPN & S* & O* & RPN* \\ \hline
       &   &   &     &    &    &      \\ \hline
       &   &   &     &    &    &      \\ \hline
       &   &   &     &    &    &      \\ \hline
       &   &   &     &    &    &      \\ \hline
       &   &   &     &    &    &      \\ \hline
\end{tabularx}%
\end{table}

Criterio adoptado: 
Se tomarán medidas de mitigación en los riesgos cuyos números de RPN sean mayores a...

Nota: los valores marcados con (*) en la tabla corresponden luego de haber aplicado la mitigación.

c) Plan de mitigación de los riesgos que originalmente excedían el RPN máximo establecido:
 
Riesgo 1: plan de mitigación (si por el RPN fuera necesario elaborar un plan de mitigación).
  Nueva asignación de S y O, con su respectiva justificación:
  - Severidad (S): mientras más severo, más alto es el número (usar números del 1 al 10).
          Justificar el motivo por el cual se asigna determinado número de severidad (S).
  - Probabilidad de ocurrencia (O): mientras más probable, más alto es el número (usar del 1 al 10).
          Justificar el motivo por el cual se asigna determinado número de (O).

Riesgo 2: plan de mitigación (si por el RPN fuera necesario elaborar un plan de mitigación).
 
Riesgo 3: plan de mitigación (si por el RPN fuera necesario elaborar un plan de mitigación).

\end{consigna}


\section{14. Gestión de la calidad}
\label{sec:calidad}

\begin{consigna}{red}
Para cada uno de los requerimientos del proyecto indique:
\begin{itemize} 
\item Req \#1: copiar acá el requerimiento.

\begin{itemize}
	\item Verificación para confirmar si se cumplió con lo requerido antes de mostrar el sistema al cliente. Detallar 
	\item Validación con el cliente para confirmar que está de acuerdo en que se cumplió con lo requerido. Detallar  
\end{itemize}

\end{itemize}

Tener en cuenta que en este contexto se pueden mencionar simulaciones, cálculos, revisión de hojas de datos, consulta con expertos, mediciones, etc.  Las acciones de verificación suelen considerar al entregable como ``caja blanca'', es decir se conoce en profundidad su funcionamiento interno.  En cambio, las acciones de validación suelen considerar al entregable como ``caja negra'', es decir, que no se conocen los detalles de su funcionamiento interno.

\end{consigna}

\section{15. Procesos de cierre}    
\label{sec:cierre}

\begin{consigna}{red}
Establecer las pautas de trabajo para realizar una reunión final de evaluación del proyecto, tal que contemple las siguientes actividades:

\begin{itemize}
	\item Pautas de trabajo que se seguirán para analizar si se respetó el Plan de Proyecto original:
	 - Indicar quién se ocupará de hacer esto y cuál será el procedimiento a aplicar. 
	\item Identificación de las técnicas y procedimientos útiles e inútiles que se emplearon, y los problemas que surgieron y cómo se solucionaron:
	 - Indicar quién se ocupará de hacer esto y cuál será el procedimiento para dejar registro.
	\item Indicar quién organizará el acto de agradecimiento a todos los interesados, y en especial al equipo de trabajo y colaboradores:
	  - Indicar esto y quién financiará los gastos correspondientes.
\end{itemize}

\end{consigna}


\end{document}
